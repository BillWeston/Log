\documentclass[11pt, oneside]{article}   	% use "amsart" instead of "article" for AMSLaTeX format
\usepackage{geometry}                		% See geometry.pdf to learn the layout options. There are lots.
\geometry{letterpaper}                   		% ... or a4paper or a5paper or ... 
%\geometry{landscape}                		% Activate for for rotated page geometry
%\usepackage[parfill]{parskip}    		% Activate to begin paragraphs with an empty line rather than an indent
\usepackage{graphicx}				% Use pdf, png, jpg, or eps§ with pdflatex; use eps in DVI mode
								% TeX will automatically convert eps --> pdf in pdflatex		
\usepackage{amssymb}

\title{Development log at CyDesign}
\author{Bill Weston}
%\date{}							% Activate to display a given date or no date

\begin{document}
\maketitle
\section{Friday 25/07/2014  Modelica Day}

Fill in some details for the models in their Modelica documentation, also(?) in the wiki.
If in doubt then {\bf here}.

\verb+(global-set-key (kbd "M-3") "#")+
\subsection{Wet Clutch}
The wet clutch did not work in pedantic because not all components of connectors were connected.\\
Investigate.\\
Fix is ...?
$$ \int_0^{\infty}\exp(-x^2) dx $$

\begin{verbatim}
(defun win-swap ()   "Swap windows using buffer-move.el"   (interactive)     
(if (null (windmove-find-other-window 'right))         
(buf-move-left)       (buf-move-right)))
\end{verbatim}

Can try a git branch and merge.
But upstairs.  Bored with here now after 2+ hours.
Do I take the charger?  Probably, then don't need to come down here in the morning.
\begin{verbatim}
touch README.md
git init
git add README.md
git commit -m "first commit"
git remote add origin https://github.com/BillWeston/ModelicaPlay.git
git push -u origin master
Push an existing repository from the command line

git remote add origin https://github.com/BillWeston/ModelicaPlay.git
git push -u origin master
\end{verbatim}

\section{Monday 28/07/2014  More Modelica}

There is defintely a bug in Dymola when trying to move between
subcomponent and supercomponent.

This is how to do the colour dynamically:
I think I need Icon( around this
       fillColor=DynamicSelect({255,255,255}, if open > 0.5 then {0,255,0} else 
                    {255,255,255}),

in controlled tank system in demos.

\verb+https://github.com/BillWeston/CppPlay.git+

It would be nice to be able to save graphical setups in Dymola.
It wouldg be nice if S-f1 S-f2 worked to switch between Modelling and Simulation.

It would be nice to practice branching and merging in git with this
while I essentially have only one file to work with.
\verb+$ git checkout -b pedanticizeWetClutch+

\subsection{Investigate torque converter}
It would be really nice work out what equations work!
\subsection{Make wet clutch test work pedantic}
Investigate the unused things from the error message.
Pete pointed out I was using an input as sensor and that if I used a
sensor it would go well.   This is indeed right.   Not {\em entirely}
sure CLI is still working, there is some error about file but that
could be to do with reaching across the directories.
\subsection{Table for controller}
\subsubsection{Create test model}
Think about interface.  Presume it is as it should be.  Think about table.  
input as from vehicle control bus, output hydro stuff..... lots of
meretricious typing in this model.

\section{Tuesday 29/07/2014  More Modelica}
[In at 0845 Out at 1815]
\subsection{Investigate torque converter}
\begin{verbatim}
From file:///Users/bill/Downloads/Modeling_Torque_Conv.pdf
Fluid density (ρ) 840 
Flow area (A) 0.0097 
Pump radius (Rp) 0.11 m 
Turbine radius (Rt) 0.066 m 
Stator radius (Rs) 0.060 m
Pump exit angle (ap) 18.01 deg
Turbine exit angle (at) -59.04 deg
Stator exit angle (as) 59.54 deg
Pump inertia (Ip) 0.092 
Turbine inertia (It) 0.026 
Stator inertia (Is) 0.012 
Fluid inertia length (Lf) 0.28 m
Shock loss coefficient (Csh) 1
Frictional loss coefficient (Cf) 0.25
Pump design constant (Sp) -0.001 
Turbine design constant (St) -0.0
0002 
Stator design constant (Ss) 0.002
\end{verbatim}
This is a useful link:\\
\verb+http://web.mit.edu/2.972/www/reports/torque_converter/torque_converter.htm+

Here is an \verb+h[]+ table
\begin{table}[h]
  \centering
  \begin{tabular}{| c | c |}
\hline
Speed Ratio & Efficiency \\
\hline
\hline
0.0 &  0.00 \\ 
0.1 &  0.25 \\ 
0.2 &  0.45 \\ 
0.3 &  0.55 \\ 
0.4 &  0.70 \\ 
0.5 &  0.80 \\ 
0.6 &  0.85 \\ 
0.7 &  0.89 \\ 
0.8 &  0.90 \\ 
0.9 &  0.91 \\ 
1.0 &  0.95 \\
\hline
  \end{tabular}
  \caption{Efficiency at Varying Speed Ratios}
  \label{tab:efficiency}
\end{table}
\begin{verbatim}
parameter Real h[11,2] = [
0.0, 0.0; 
0.1, 0.25; 
0.2, 0.45; 
0.3, 0.55; 
0.4, 0.70; 
0.5, 0.8; 
0.6, 0.85; 
0.7, 0.89; 
0.8, 0.9; 
0.9, 0.91; 
1.0, 1.0];
\end{verbatim}
Here is an \verb+r[]+ table
parameter Real r[11,2] = [0.0, 2.2; 0.1, 2.01; 0.2, 1.87; 0.3, 1.80; 0.4, 1.65; 0.5, 1.6; 0.6, 1.50; 0.7, 1.30; 0.8, 1.2; 0.9, 1.00; 1.0, 1.0];

I feared I may have lost some stuff from Tuesday due to bad copying.
Never mind, I worked on the Torque Converter and got anaconda python
working.

\section{Wednesday 30/07/2014}
[in at 0900.  Out at 17.30]

Torque converter.
/Users/bill/Library/Preferences/Aquamacs
Emacs/Preferences.el/Users/bill/Library/Preferences/Aquamacs

Did Torque Converter without pictures.
Improve test harness so as to have a big downstream inertia and a
damping

Roughly speaking gear ratio of 4, final drive of 4 so r is about 15
$$
\alpha = {r^2\over MR^2} \tau
$$
$$
J = {MR^2 \over r^2}
$$
$J$ of rest of driveline is order of 1.
mass is about $1500kg$, radius $R$ about a foot, or $ \sqrt{10} $ if
you like.

This is me using the wedge.  I am not sure it is particularly useful.

\section{Thursday 31/07/2014}
OK really getting short of time now.
New plan is to do a Kinematic model and forget about hydraulics for
now.
Get something working.
Also go buy lunch
The new model has simple tablular manifold/controller which works in
unit test (be sure to use built in msl).

%%%%%%%%%%%%%%%%%%%%%%%%%%%%%%%%%%%%%%%%%%%%%%%%%%%
\section{Friday 01/08/2014}
Goal is to get Kinetic model running.

\section{General Tasks and Points}
\label{sec:general-tasks-points}

\begin{itemize}
 \item Put into Components package:
   \begin{itemize}
   \item CyTransmission.Examples.ZF9HPKinematic 
   \item CyTransmission.Examples.ZF9HPKvil 
   \end{itemize} 
\item Add Test Icon to ZF9HP Test. 
\item Oil for  ZF9HP Test. 
\end{itemize}

\subsection{Kinetic Manifold}
\label{sec:kinetic-manifold}
Got it running.  Check by hand that the outputs are correct.  {\bf The
outputs are correct.}
Think about asserts?
Pete points out that it is not working in CyCLI because can only add
one level at a time access into an otherwise non-existent bus.
(A Dymola vs Pure Modelica difference).

\subsection{ZF9HP Test}
\label{sec:zf9hp}
Got to battle with buses.   Will tidy first, will commit first.
And then afterwards to right repo on server with
\verb+git push -u origin KinematicModel+
Or whatabout
\verb+git push --all -u+



\end{document}  
